\documentclass[
	12pt,				% tamanho da fonte
	oneside,			% para impressão em recto e verso. Oposto a oneside
	a4paper,			% tamanho do papel. 
	english,			% idioma adicional para hifenização
	brazil,				% o último idioma é o principal do documento
	]{abntex2}

% ---
% Pacotes fundamentais 
% ---
\usepackage{lmodern}			% Usa a fonte Latin Modern
\usepackage[T1]{fontenc}		% Selecao de codigos de fonte.
\usepackage[utf8]{inputenc}		% Codificacao do documento (conversão automática dos acentos)
\usepackage{indentfirst}		% Indenta o primeiro parágrafo de cada seção.
\usepackage{color}				% Controle das cores
\usepackage{graphicx}			% Inclusão de gráficos
\usepackage{microtype} 			% para melhorias de justificação
\usepackage{multicol}
\usepackage{multirow}
\usepackage[brazilian,hyperpageref]{backref}	 % Paginas com as citações na bibl
\usepackage[alf]{abntex2cite}	% Citações padrão ABNT
\usepackage{float}
% --- 
% CONFIGURAÇÕES DE PACOTES
% --- 

% ---
% Configurações do pacote backref
% Usado sem a opção hyperpageref de backref
\renewcommand{\backrefpagesname}{Citado na(s) página(s):~}
% Texto padrão antes do número das páginas
\renewcommand{\backref}{}
% Define os textos da citação
\renewcommand*{\backrefalt}[4]{
	\ifcase #1 %
		Nenhuma citação no texto.%
	\or
		Citado na página #2.%
	\else
		Citado #1 vezes nas páginas #2.%
	\fi}%
% ---

% ---
% Informações de dados para CAPA e FOLHA DE ROSTO
% ---
\titulo{Prática 11\\Laboratório de AEDs}
\autor{Pedro Inácio Rodrigues Pontes}
\local{Belo Horizonte, Brasil}
\data{2024}
\instituicao{%
  Universidade Federal de Minas Gerais
  \par
  Colégio Técnico
  \par
  Curso Técnico em Desenvolvimento de Sistemas}

\definecolor{blue}{RGB}{41,5,195}

\makeatletter
\hypersetup{
     	%pagebackref=true,
		pdftitle={\@title}, 
		pdfauthor={\@author},
    	pdfsubject={\imprimirpreambulo},
		colorlinks=true,       		% false: boxed links; true: colored links
    	linkcolor=blue,          	% color of internal links
    	citecolor=blue,        		% color of links to bibliography
    	filecolor=magenta,      		% color of file links
		urlcolor=blue,
		bookmarksdepth=4
}
\makeatother

\renewcommand{\thesection}{\arabic{section}}
\setlength{\parindent}{1.3cm}
\setlength{\parskip}{0.2cm} 

\makeindex


\begin{document}

\selectlanguage{brazil}
\frenchspacing 

\imprimircapa

{
\ABNTEXchapterfont

\textual

% ----------------------------------------------------------
% Introdução (exemplo de capítulo sem numeração, mas presente no Sumário)
% ----------------------------------------------------------
\section{Introdução}

O trabalho em questão trata-se de uma virtualização do jogo Torre de Hanoi. A sua implementação virtual foi feita utilizando-se da linguagem de programação Processing Java. As regras do jogo são:

\begin{itemize}
\item Apenas um disco pode ser movido por vez
\item Somente o disco superior de uma pilha pode ser transferido para o topo de outra pilha ou para uma barra vazia
\item Discos maiores não podem ser empilhados sobre discos menores
\end{itemize}

O objetivo dessa prática foi aplicar os conhecimentos sobre Stacks, adquiridos na aula nº 11 de AEDS. Percebe-se que a torre tem um funcionamento bastane similar a uma stack, pois apenas o elemento mais recente/superior da torre pode ser movido, enquanto mais antigos/inferiores só são movidos após todos os mais recentes/superiores que eles serem retirados.

\section{Desenvolvimento}

Para criar essa virtualização, foram utilizadas 4 classes: 
\begin{itemize}
 \item Timer
 \item Pillar
 \item Disc
 \item Main
\end{itemize}

 \subsection{Timer}
 Registra o tempo. Visualiazação do seu código principal:
  \begin{itshape}
 \begin{verbatim}
  public void update(){
    iterations ++;
    if (iterations % framerate == 0){
      seconds ++;
    }
    if (seconds == 60){
      seconds = 0;
      minutes ++;
    }
    formatedTime = String.format("%02d:%02d", minutes, seconds); 
  }
 \end{verbatim}
 \end{itshape}
 \subsection{Pillar}
 Cria os pilares sobre os quais os discos podem ser empilhados. O construtor da classe foi criado para ela ser inicalizada por meio de iteração e guardada num array de classe - tal coisa simplifica e enxuta os códigos que usam atributos e métodos dos 3 pilares. Como no jogo existem tres pilares i < 3 na iteração. Visão do construtor:

\begin{itshape}
\begin{verbatim}
public Pillar (int i) {
    this.Width = 10;
    this.Height = 490;
    this.xPos = 142.5 + i * (142.5 + Width);
    this.yPos = 100;
  }
\end{verbatim}
\end{itshape}

O cálculo da posição x dos pilares é feito a partir do cálculo presente acima baseado nos valores de i. No fim, serão criados três pilares igualmente espaçados. 

Há um método show() para exibir o pilar na interface gráfica

\subsection{Disc}
Cria os discos usados na torre. Segue o mesmo princípio de inicialização por meio de iteração sob algum comando \textit{ex. for} e armazenamento em array de classe. Visão do construtor:

\begin{itshape}
\begin{verbatim}
public Disc (int i, float size){
    this.Width = size;
    this.Height = Width/3;
    this.Width -= i*10;
    this.xPos = (width/2 - Width - 5)/2;
    this.yPos = height - (Height * (i + 1)) - 10;
    this.color1 = random(255);
    this.color2 = random(255);
    this.color3 = random(255);
  }
\end{verbatim}
\end{itshape}
Aqui, são inicializadas randomicamente as cores e são feitos diversos cáculos para definir \textit{Width, Height, xPos, yPos}, tendo como base os parâmetros i e size. size é utilizada para randomizar as dimensões dos discos, mais especificamente do disco base, o qual influencia o valor de todos os outros. Se size fosse definida dentro do construtor, todos os discos teriam tamanhos randomizados, fazendo com que o ordenamento de maior embaixo e menor em cima seja perdido.

Também é feito um construtor para quando é necessário apenas inicializar a classe, sem inicializar seus valores, exceto Width:

\begin{itshape}
\begin{verbatim}
public Disc(float size){
    this.Width = size;
}
\end{verbatim}
\end{itshape}
É inicializado Width para poder manter um ordenamento desses discos com atributos não inicializados a partir de seu tamanho.

Há um método show() para exibir o objeto disc na interface gráfica

\subsection{Main}
Classe principal, possui 9 funções. As quais são descritas nas próximas sub subseções Globalmente, são inicializadas 7 variáveis, das quais 6 desempenham papel essencial na dinâmica do jogo:

\begin{itemize}
    \item int move
    \item int referencedDisc = 0
    \item int totalDiscs = 5
    \item float size = random(110,135)
    \item boolean win = false
    \item boolean menu = true
    \item boolean initialize = true
    \end{itemize}

É instanciada a classe Timer, o array da classe Pillar e o array de tower, a qual é uma stack que contém discos.
\subsubsection{void setup()}
Função Principal. Inicializa o tamanho da interface gráfica do Processing onde será executado o código, as imagens e fontes, e é feito um for que instancia os pilares e towers:

\begin{itshape}
\begin{verbatim}
for (int i = 0; i < 3; i++) {
  tower[i] = new Stack<>();
  pillar[i] = new Pillar(i);
\end{verbatim}
\end{itshape}
\subsubsection{void draw()}
Função principal. Chama o restante das funções. O código é autoexplicativo:

\begin{itshape}
\begin{verbatim}
void draw() {
  if (menu){
    initialMenu(); 
  }
  else{
    if(win == false){  
      game();
    }
    else{
      finalMenu();
    }
  }

  //Verify if the player wins
  if (tower[1].size() == totalDiscs || tower[2].size() == totalDiscs){
    win = true;  
  }
}

\end{verbatim}
\end{itshape}
\subsubsection{void initialMenu()}
Carrega o menu inicial
\subsubsection{void finaMenu()}
Carrega o menu final
\subsubsection{void game()}
Inicializa os discos de torre. Eles são inicializados fora da setup() por haver a necessidade de esperar o jogador escolher o número desejado de discos para a partida. Fazer essa inicialização uma única vez exige um código um pouco mais complexo. Segue:

\begin{itshape}
\begin{verbatim}
if (initialize) {
  for (int i = 0; i < totalDiscs; i++) {
    tower[0].push(new Disc(i,size));
    }
  initialize = false;
}
\end{verbatim}
\end{itshape}

O timer é ativado até o fim da partida. São exibidos os pilares e discos. Se um disco for clicado pelo mouse, tal seguirá ele. Isso é feito pelo seguinte código, que tem relação direta com a função moveDisc():

\begin{itshape}
\begin{verbatim}
Disc d = tower[referencedDisc].peek();
d.show();
      
//To the disc follows the mouse after it is mousePressed
if (move == 1) {
    d.xPos = mouseX - d.Width/2;
    d.yPos = mouseY - d.Height/2;
} 
\end{verbatim}
\end{itshape}
\subsubsection{void moveDisc()}
É chamada dentro de mousePressed(). Códigos que possibilitam o disco seguir o mouse após clicado e ir para certa torre quando há um clique posterior na sua área. Segue o código: 

\begin{itshape}
\begin{verbatim}
if (initialize == false){
  Disc d = tower[referencedDisc].peek(); //Disc that is been referenced
  Disc[] disc = new Disc[3];

  //Create a instance of peek disc for each tower
  for (int i = 0; i < 3; i ++){
    if (!tower[i].isEmpty()){
      disc[i] = tower[i].peek();
    }
    else{
      disc[i] = new Disc(size);
    }
  }

  //To the disc follows the mouse after it is mousePressed
  for (int i =0; i < 3; i ++){
    if (mouseX >= disc[i].xPos && mouseX <= disc[i].xPos + disc[i].Width &&
        mouseY >= disc[i].yPos && mouseY <= disc[i].yPos + disc[i].Height) {
      move = -move;
      referencedDisc = i;
      d = tower[referencedDisc].peek(); 
    }
  }

  //Repositions the disc after a click in a allowed local
  if (mouseX >= d.xPos && mouseX <= d.xPos + d.Width &&
      mouseY >= d.yPos && mouseY <= d.yPos + d.Height) {
    for (int i = 0; i < 3; i ++){
      if (move == -1 && mouseX >= pillar[i].xPos -30 && mouseX <= pillar[i].xPos + pillar[i].Width + 30 &&
          mouseY >= pillar[i].yPos && mouseY <= pillar[i].yPos + pillar[i].Height && d.Width <= disc[i].Width){
        tower[referencedDisc].pop();
        d.xPos = pillar[i].xPos - d.Width/2 + 5;
        d.yPos = height - (d.Height * (tower[i].size() + 1)) - 10;  
        disc[i] = tower[i].push(d); 
        referencedDisc = i;
      }
    }
  }
}
\end{verbatim}
\end{itshape}

O código abaixo do comentário \textit{//Repostions the disc after a click in a allowed local} é funadamental para todo o programa, pois é ele que administra o algoritimo que possibilita a movimentação de um disco de torre para outra, adminstrando os stacks envolvidos. Em grau de importância, ele está em primeiro lugar. Nesse código, basicamente é retirado o disco da torre de que saiu, setada sua posição x para o meio da torre escolhido para ser colocado e sua posição y para acima do disco mais superior da torre escolhida, adicionado o disco à torre que foi colocado e atualizado o referencedDisc para o código ser executado corretamente.

\subsubsection{void changeTotalDiscs()}
Chamada dentro de mousePressed(). Gere o total de discos e os botões para aumentar ou diminuí-los. Aparece apenas no menu inicial.
\subsubsection{void startGame()}
Chamada dentro de mousePressed(). Gere o botão de Play e causa o subsequente iniciar do jogo após ele ser clicado. Aparece apenas no menu inicial.
\subsubsection{void mousePressed()}
Proporciona que moveDiscs(), changeTotalDiscs() e startGame() sejam chamadas apenas quando o mouse seja clicado. Vale a pena lembrar que essas funções também tem internamente condições para poderem ser executadas.

\section{Resultados}

\begin{figure}[H]
  \centering
  \includegraphics[width=\linewidth]{imagem1.png}
  \caption{Menu Inicial}
\end{figure}

\begin{figure}[H]
  \centering
  \includegraphics[width=\linewidth]{imagem2.png}
  \caption{Botão para aumentar tamanho da torre funcionando}
\end{figure}

\begin{figure}[H]
  \centering
  \includegraphics[width=\linewidth]{imagem3.png}
  \caption{Botão para diminuir tamanho da torre funcionando}
\end{figure}

\begin{figure}[H]
  \centering
  \includegraphics[width=\linewidth]{imagem4.png}
  \caption{Botão play funcionando}
\end{figure}

\begin{figure}[H]
  \centering
  \includegraphics[width=\linewidth]{imagem5.png}
  \caption{Visão inicial do jogo}
\end{figure}

\begin{figure}[H]
  \centering
  \includegraphics[width=\linewidth]{imagem6.png}
  \caption{Discos movimentados}
  \label{fig:carousel}
\end{figure}

\begin{figure}[H]
  \centering
  \includegraphics[width=\linewidth]{imagem7.png}
  \caption{Torre quase ordenada}
\end{figure}

\begin{figure}[H]
  \centering
  \includegraphics[width=\linewidth]{imagem8.png}
  \caption{Torre ordenada (um instante antes da aparição do menu final)}
\end{figure}

\begin{figure}[H]
  \centering
  \includegraphics[width=\linewidth]{imagem9.png}
  \caption{Menu Final}
\end{figure}

\section{Conclusão}


\begin{itemize}
    \item Todos os resultados estão de acordo com o esperado. O jogo foi criado corretamente implementando suas regras e stacks para armazenar os discos de cada torre. O aprendizado prático em stacks objetivado também foi alcançado.
    \item A principal dificuldade durante a execução do programa foi a criação da função \textit{moveDiscs()} para gerenciar as torres e discos. Houveram diversas tentativas falhas até alcançar uma abstração que funcionasse corretamente. Foi revolucionário para o funcionamento correto da função e código a criação de um index \textit{referencedDisc} e as 3 instâncias de peek disc para cada torre. Outra dificuldade digna de ser mencionada foi criar o construtor de Disc planejado para uso dentro de um loop for, pois os cálculos das dimensões e posições do disco se tornaram muito complexos, por falta de planejamento e má implementação, principalmente.
    \item Diante da realização desse experimento os maiores aprendizados foram da necessidade de se planejar um código antes de sua criação - o que teria evitado a dificuldade em criar \textit{moveDiscs()} e os calcúlos do construtor Disc -, o do uso máximo de 1 hora e 30 minutos consecutivos para períodos focados de trabalho (após isso tudo se torna confuso e a irritabilidade aumenta) e a melhora na capacidade de abstração e uso de stacks.
\end{itemize}


\end{document}
